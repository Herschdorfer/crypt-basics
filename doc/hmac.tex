\chapter{HMAC}

\section{Grundlagen}
Ein \gls{hmac} ist ein kryptographischer Code, der eine Nachricht und einen Schlüssel verwendet, um die Integrität und Authentizität der Nachricht zu gewährleisten. Der \gls{hmac} Algorithmus basiert in der Regel auf einem Hash-Algorithmus, wie z.B. \gls{sha} \cite{rfc2104}.

\section{Implementierung}
Die Klasse \texttt{HMAC} implementiert den \gls{hmac} Algorithmus \glqq HmacSHA256\grqq.
Der Constructor der Klasse \texttt{HMAC} erwartet ein \texttt{byte[]} Array als Schlüssel.
Die Klasse bietet zwei Funktionen an: \texttt{calc} und \texttt{verify}.

\subsection{Methode: calc}
Die Methode \texttt{calc} erwartet ein \texttt{byte[]} Array als Nachricht und gibt ein \texttt{byte[]} Array als \gls{hmac} zurück.

\subsection{Methode: verify}
Die Methode \texttt{verify} erwartet ein \texttt{byte[]} Array als Nachricht und ein \texttt{byte[]} Array als \gls{hmac}.
Die Methode gibt \texttt{true} zurück, wenn die \gls{hmac} der Nachricht entspricht, ansonsten \texttt{false}.

