\chapter{Ver-/Entschlüsselung sowie Schlüsselgenerierung mit RSA}

\section{Grundlagen}
\gls{rsa} ist ein asymmetrisches Verschlüsselungsverfahren, das 1977 von Ron Rivest, Adi Shamir und Leonard Adleman entwickelt wurde. Es basiert auf der Schwierigkeit des Faktorisierens großer Zahlen \cite{rsaPaper}.

\section{Implementierung}
Die \gls{rsa} Implementierung ist in zwei Klassen aufgeteilt.
\begin{itemize}
	\item Die Klasse \texttt{RSAKeyGen} ist für die Schlüsselerzeugung zuständig.
	\item Die Klasse \texttt{RSA} ist für die Verschlüsselung und Entschlüsselung zuständig.
\end{itemize}

\subsection{Klasse: RSAKeyGen}
Der Constructor der Klasse \texttt{RSAKeyGen} erzeugt $n = p*q$ und $\Phi(n) = (p-1)*(q-1)$ und $d = e^{-1} \mod \Phi(n)$ \cite{rfc8017}. Danach werden $p$, $q$ und $\Phi(n)$ verworfen. Der Exponent $e$ wird auf Feld der Klasse \texttt{RSAKeyGen} gespeichert.

\subsection{Klasse: \gls{rsa}}
Der Constructor der Klasse \texttt{\gls{rsa}} erzeugt ein Objekt der Klasse \texttt{RSAKeyGen} und speichert die Instanz auf einem Feld.

\subsubsection{Verschlüsselung}
Die Verschlüsselung erfolgt durch die Methode \texttt{encrypt}. Das übergebene \texttt{char} Array wird einzeln wird mit dem öffentlichen Schlüssel potenziert \cite{rfc8017} und und in einem Array vom Typ \texttt{BigInteger} zurückgegeben.

\subsubsection{Entschlüsselung}
Die Entschlüsselung erfolgt analog zu Verschlüsselung, jedoch wird mit dem privaten Schlüssel potenziert. Eingabe ist ein Array vom Typ \texttt{BigInteger} und die Ausgabe ist ein \texttt{char} Array.

\section{Parameter}
Die Parameter der Klasse \texttt{\gls{rsa}} sind vom Typ \texttt{BigInteger} und können somit Bitgrößen wie 1024 Bit, 2048 Bit oder mehr verarbeiten. Der Exponent $e$ kann entweder 65537 oder beliebig sein, je nach verwendetem Constructor. 65537 ist ein Standardwert für $e$ und wird in der Praxis oft verwendet.

Die Verschlüsselung ist nicht optimal implementiert, da jeder Character in einem String einzeln verschlüsselt wird. Damit wird jeder Character mit dem öffentlichen Schlüssel potenziert wird und damit die Bitgröße von 8 teilweise deutlich überschritten. Die Verschlüsselung ist daher nicht effizient und kann bei großen Texten zu Problemen führen.